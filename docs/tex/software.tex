\chapter{分析软件介绍}

\section{配置文件-crate.json}

\section{ROOT输出文件格式}

\begin{table}
	\centering
		\begin{tabular}{|l|p{2cm}|l|p{6cm}|}
			\hline
			Branch Name &  Type & Value & Meaning \\ 
			\hline
			\texttt{event\_flag} &  \texttt{Char\_t}(B) & 1,2,3 & 代表该Event格式是否正确,1是正常情况,2是有Header无Trailer(一般是文件最后一个不完整的事件),3是Header中的EventID不等于Trailer中的EventID \\ 
			\hline
			\texttt{event\_id} & \texttt{Int\_t}(B) & 12 bits & 代表该Event的事件号(来自Master TDC),是直接从数据中解码出来的。由于Group Header和Trailer中都包含EventID,若\texttt{event\_flag=3},则该字段是来自Header\\ 
			\hline
			\texttt{bunch\_id} & \texttt{Int\_t} & 12 bits & 直接从Group Header中解码得到(来自Master TDC)。\\
			\hline
			\texttt{leading\_raw} & \texttt{std::map<Uint\_t,std::vector<int>} & 19 bits & MWDC板的高精度TOT后沿时间测量值,key是面板上的通道编号\\
			\hline
			\texttt{trailing\_raw} & \texttt{std::map<Uint\_t,std::vector<int>} & 19 bits & MWDC板的高精度TOT前沿时间测量值,key是面板上的通道编号 \\
			\hline
			\texttt{time\_leading\_raw} & \texttt{std::map<Uint\_t,std::vector<int>} & 21 bits & TOF板甚高精度TOT前沿时间测量值,key是面板上的通道编号 \\
			\hline
			\texttt{time\_trailing\_raw} & \texttt{std::map<Uint\_t,std::vector<int>} & 21 bits & TOF板的甚高精度TOT后沿时间测量值,key是面板上的通道编号 \\
			\hline
			\texttt{tot\_leading\_raw} & \texttt{std::map<Uint\_t,std::vector<int>} & 19 bits & TOF板的高精度TOT前沿时间测量值,key是面板上的通道编号 \\
			\hline
			\texttt{tot\_trailing\_raw} & \texttt{std::map<Uint\_t,std::vector<int>} & 19 bits & TOF板的高精度TOT后沿时间测量值,key是面板上的通道编号 \\
			\hline
		\end{tabular}
		\caption{每块MWDC卡/TOF卡解码后得到的TTree格式}
		\label{tbl:raw_format}
\end{table}