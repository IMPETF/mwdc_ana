\documentclass[a4paper,12pt]{ctexrep}
\usepackage{float}
\usepackage{subcaption}
\usepackage{amsmath}
\usepackage{color}

\begin{document}
	% top matter
	\title{\textbf{MWDC宇宙线标定测试平台用户指南}}
	\author{周勇,唐述文\\
	次级束物理研究组\\
	中国科学院近代物理研究所\\
	\texttt{yong@impcas.ac.cn, zyong06@gmail.com}}

	\maketitle

	% article and report can have an abstract
	\begin{abstract}
	MWDC宇宙线标定测试平台是一套完整的宇宙线刻度系统,它由以下几个部分组成:大真空靶室及其辅助真空设备,径迹探测器分系统,触发探测器分系统,数据获取分系统(DAQ)以及离线分析软件。

	该测试平台最初是基于暗物质粒子探测卫星(DAMPE)塑闪阵列探测器的地面宇宙线刻度而设计和搭建的,因此其主体框架是一套大真空靶室以及相应的辅助真空设备。测试时,被测探测器放置在真空靶室内部。

	径迹探测器分系统由两个多丝漂移室(MWDC)构成,分别放置在靶室上部和下部,用于确定入射宇宙线的径迹。根据测得的径迹,可以反推出其在被测探测器上的击中位置,从而准确得到探测器的位置响应。

	触发探测器分系统由两个大面积的塑闪板探测器构成,分别放置在上MWDC的上部和下MWDC的下部。它用于标记入射宇宙线事例,为整个平台提供快触发信号和起始时间信号。

	离线分析软件基于ROOT和C++,用于原始数据的解码,数据质量的校验以及数据分析。
	\end{abstract}

	\tableofcontents
	\listoffigures
	\listoftables

	\chapter{探测器部分}

\section{整体布局与坐标系统}


\section{探测器结构}
\subsection{多丝漂移室}

\subsection{大面积塑闪板}

\subsection{探测器编号}

\section{真空靶室与相关设备}



	\chapter{电子学部分}
\section{PXI获取系统与HPTDC简介}

\subsection{HPTDC基本参数}
HPTDC基本参数如表\ref{tbl:hptdc_parameters}所示:
\begin{table}
	\centering
		\begin{tabular}{|l|l|l|p{6cm}|}
			\hline
			参数 & 意义 & 数值 & 备注 \\ 
			\hline
			Bunch ID & 触发信号的时间戳 & 用12 bits存储,单位25ns,范围102.4us &  \\ 
			\hline
			高精度测量的精度 & High Precision Mode & 19 bits, 单位$25/256 ns\simeq 100 ps$,范围51.2us &  \\ 
			\hline
			超高精度测量的精度 & Very High Precision Mode & 21 bits, 单位$25/256/4 ns \simeq 25 ps$ ,范围51.2us & \\
			\hline
			Coarse Time Counter & 用于拓展时间测量量程 & 时钟源来自PLL(40,160,320Hz),共15 bits & 40Hz是使用低12位,160Hz使用低14位,320Hz使用全部15位 \\
			\hline
			DLL & 用于精细时间测量 & 时钟源来自PLL(40,160,320Hz),将一个周期时钟分为32份,等效于5 bits & \\
			\hline
		\end{tabular}
		\caption{HPTDC基本性能参数}
		\label{tbl:hptdc_parameters}
\end{table}

\subsection{HPTDC使用要点}

\section{MWDC读出电子学}

\subsection{通道对应关系}

\section{塑闪板读出电子学}

\subsection{通道对应关系}

\section{触发板与时钟板}

\section{数据格式}

\subsection{Ungrouped格式}
在2015年2、3月份间的测试过程中,原先设计的事件打包逻辑有bug存在,因此正式的宇宙线测试中使用了Ungrouped格式,即使用HPTDC原始的打包格式。具体如下:
\begin{itemize}
	\item 每块获取卡对应一个独立的原始数据文件,文件具有相同的前缀,和后缀“atl”,并使用获取卡所在的槽位编号。
	\item 数据逐事件存储。每个Event的数据由一个Group Header和一个Group Trailer组成的数据段组成;Header和Trailer中间是各块HPTDC的测量数据,且只是各通道的测量数据,而不包含HPTDC的Header和Trailer。每个word的具体格式参考HPTDC说明书。
	\item 对于MWDC来说,获取板上四片HPTDC从0-3编号,读取顺序为3-2-1-0;\textbf{出于未知的原因,MWDC的信号前沿是原始数据中的Trailing Time,而信号后沿是原始数据中的Leading Time。}。
	\item 对于TOF获取板来说,其信号前后沿与原始数据中的Leading/Trailing Time是一致的。
	\item Bunch ID的决定值是与trigger latency相关的,当所有获取卡的trigger latency一致时,同一事件的Bunch ID也应该是一致的;若trigger latency不一致,则Bunch ID的差值就是Trigger Latency的大小。这是因为trigger time tag是要减去trigger latency后才存入Trigger FIFO中的。
\end{itemize}

已知问题:
\begin{enumerate}
	\item 对于单个事例来讲,MWDC获取板上的最大读出数据长度为$256\times4=1024$个word(根据HPTDC的Readout FIFO存储深度计算)。实际测试中发现,偶尔会出现大事例,此时数据长度超过1024个word,但不影响数据格式。
	\item 在正确使用触发板的条件下,不同获取卡间的bunch id基本能够对齐,偶尔会出现$\pm 1$的浮动,但会恢复。
	\item 在正确使用触发板的条件下,每块获取卡内的event id是连续的;基本都能从0开始计数,偶尔会有若干获取卡可能不从0开始。
	\item Group Trailer中的word count是不准确的
	\item Group Header和Group Trailer中的event id是一致的。
\end{enumerate}





	\chapter{分析软件介绍}

\section{配置文件-crate.json}

\section{ROOT输出文件格式}

\begin{table}
	\centering
		\begin{tabular}{|l|p{2cm}|l|p{6cm}|}
			\hline
			Branch Name &  Type & Value & Meaning \\ 
			\hline
			\texttt{event\_flag} &  \texttt{Char\_t}(B) & 1,2,3 & 代表该Event格式是否正确,1是正常情况,2是有Header无Trailer(一般是文件最后一个不完整的事件),3是Header中的EventID不等于Trailer中的EventID \\ 
			\hline
			\texttt{event\_id} & \texttt{Int\_t}(B) & 12 bits & 代表该Event的事件号,是直接从数据中解码出来的。由于Group Header和Trailer中都包含EventID,若\texttt{event\_flag=3},则该字段是来自Header\\ 
			\hline
			\texttt{bunch\_id} & \texttt{Int\_t} & 12 bits & 直接从Group Header中解码得到。\\
			\hline
			\texttt{leadin\_raw} & \texttt{std::map<Uint\_t,std::vector<int>} & 19 bits & MWDC/TOF板的高精度TOT前沿时间测量值,key是面板上的通道编号\\
			\hline
			\texttt{trailing\_raw} & \texttt{std::map<Uint\_t,std::vector<int>} & 19 bits & MWDC板的高精度TOT前沿时间测量值,key是面板上的通道编号 \\
			\hline
			\texttt{leadin\_raw} & \texttt{std::map<Uint\_t,std::vector<int>} & 19 bits & MWDC板的高精度TOT前沿时间测量值,key是面板上的通道编号\\
			\hline
			\texttt{time\_leading\_raw} & \texttt{std::map<Uint\_t,std::vector<int>} & 21 bits & TOF板甚高精度TOT前沿时间测量值,key是面板上的通道编号 \\
			\hline
			\texttt{time\_trailing\_raw} & \texttt{std::map<Uint\_t,std::vector<int>} & 21 bits & TOF板的甚高精度TOT前沿时间测量值,key是面板上的通道编号 \\
			\hline
			\texttt{tot\_leading\_raw} & \texttt{std::map<Uint\_t,std::vector<int>} & 19 bits & TOF板的高精度TOT前沿时间测量值,key是面板上的通道编号 \\
			\hline
			\texttt{tot\_trailing\_raw} & \texttt{std::map<Uint\_t,std::vector<int>} & 19 bits & TOF板的高精度TOT前沿时间测量值,key是面板上的通道编号 \\
			\hline
		\end{tabular}
		\caption{每块MWDC卡/TOF卡解码后得到的TTree格式}
		\label{tbl:raw_format}
\end{table}

	\chapter{MWDC的自身刻度}


	\appendix
	\chapter{2015年2、3月份间的宇宙线测试数据分析}

\section{触发时间间隔分布}
\subsection{问题描述}
一般认为宇宙线事例的时间间隔分布满足指数衰减规律,即
\begin{equation}
	P(t) = \frac{1}{T_{interval}}e^{-t/T_{interval}}
\end{equation}
其中,$T_interval$是宇宙线事例的平均时间间隔。

\section{获取卡间的时间同步(Time and event synchronization among different DAQ cards)}

\section{单通道的多击中数(Multi-hits in a single channel)}

\section{MWDC单个丝面的击中多重数(Multiplicity in a single MWDC wireplane}

\end{document}