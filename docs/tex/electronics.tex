\chapter{电子学部分}
\section{PXI获取系统与HPTDC简介}

\subsection{HPTDC基本参数}
HPTDC基本参数如表\ref{tbl:hptdc_parameters}所示:
\begin{table}
	\centering
		\begin{tabular}{|l|l|l|p{6cm}|}
			\hline
			参数 & 意义 & 数值 & 备注 \\ 
			\hline
			Bunch ID & 触发信号的时间戳 & 用12 bits存储,单位25ns,范围102.4us &  \\ 
			\hline
			高精度测量的精度 & High Precision Mode & 19 bits, 单位$25/256 ns\simeq 100 ps$,范围51.2us &  \\ 
			\hline
			超高精度测量的精度 & Very High Precision Mode & 21 bits, 单位$25/256/4 ns \simeq 25 ps$ ,范围51.2us & \\
			\hline
			Coarse Time Counter & 用于拓展时间测量量程 & 时钟源来自PLL(40,160,320Hz),共15 bits & 40Hz是使用低12位,160Hz使用低14位,320Hz使用全部15位 \\
			\hline
			DLL & 用于精细时间测量 & 时钟源来自PLL(40,160,320Hz),将一个周期时钟分为32份,等效于5 bits & \\
			\hline
		\end{tabular}
		\caption{HPTDC基本性能参数}
		\label{tbl:hptdc_parameters}
\end{table}

\subsection{HPTDC使用要点}

\section{MWDC读出电子学}

\subsection{通道对应关系}

\section{塑闪板读出电子学}

\subsection{通道对应关系}

\section{触发板与时钟板}

\section{数据格式}

\subsection{Ungrouped格式}
在2015年2、3月份间的测试过程中,原先设计的事件打包逻辑有bug存在,因此正式的宇宙线测试中使用了Ungrouped格式,即使用HPTDC原始的打包格式。具体如下:
\begin{itemize}
	\item 每块获取卡对应一个独立的原始数据文件,文件具有相同的前缀,和后缀“atl”,并使用获取卡所在的槽位编号。
	\item 数据逐事件存储。每个Event的数据由一个Group Header和一个Group Trailer组成的数据段组成;Header和Trailer中间是各块HPTDC的测量数据,且只是各通道的测量数据,而不包含HPTDC的Header和Trailer。每个word的具体格式参考HPTDC说明书。
	\item 对于MWDC来说,获取板上四片HPTDC从0-3编号,读取顺序为3-2-1-0;\textbf{出于未知的原因,MWDC的信号前沿是原始数据中的Trailing Time,而信号后沿是原始数据中的Leading Time。}。
	\item 对于TOF获取板来说,其信号前后沿与原始数据中的Leading/Trailing Time是一致的。
	\item Bunch ID的决定值是与trigger latency相关的,当所有获取卡的trigger latency一致时,同一事件的Bunch ID也应该是一致的;若trigger latency不一致,则Bunch ID的差值就是Trigger Latency的大小。这是因为trigger time tag是要减去trigger latency后才存入Trigger FIFO中的。
\end{itemize}

已知问题:
\begin{enumerate}
	\item 对于单个事例来讲,MWDC获取板上的最大读出数据长度为$256\times4=1024$个word(根据HPTDC的Readout FIFO存储深度计算)。实际测试中发现,偶尔会出现大事例,此时数据长度超过1024个word,但不影响数据格式。
	\item 在正确使用触发板的条件下,不同获取卡间的bunch id基本能够对齐,偶尔会出现$\pm 1$的浮动,但会恢复。
	\item 在正确使用触发板的条件下,每块获取卡内的event id是连续的;基本都能从0开始计数,偶尔会有若干获取卡可能不从0开始。
	\item Group Trailer中的word count是不准确的
	\item Group Header和Group Trailer中的event id是一致的。
\end{enumerate}



